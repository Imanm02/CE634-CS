\section*{سوال ۵}

پس از راه‌اندازی آزادراه تهران-شمال، وزارت راه و شهرسازی قصد دارد بین استفاده از باجه‌ی اتوماتیک و یا باجه‌ی سنتی برای دریافت عوارضی یکی را انتخاب کند. هزینه‌ی باجه‌ی اتوماتیک دو برابر باجه‌های سنتی است و بودجه این وزارت‌خانه برای این پروژه برابر هزینه‌ی یک باجه‌ی اتوماتیک است. بنابراین می‌تواند یا یک باجه‌ی اتوماتیک و یا دو باجه‌ی سنتی قرار دهد. همچنین فرض کنید خودروها را با یک فرآیند پواسون با نرخ 1000 نفر در ساعت مدل کنیم که به عوارضی یک شهر می‌رسند و در یک صف طولانی قرار می‌گیرند. با توجه به زمان انتظار مورد انتظار، کدام انتخاب برای وزارت راه و شهرسازی بهتر است؟

\textbf{الف)}
این صف توسط یک باجه ی خودکار با نرخ 1200 نفر در ساعت خدمت رسانی می شود. همچنین فرض کنید که زمان‌های سرویس‌دهی از توزیع نمایی پیروی کنند و استراتژی سرویس‌دهی FIFO است. پیدا کنید زمان انتظار مورد انتظار در حالت پایدار 
\lr{(Steady-state expected waiting time)}.

\textbf{ب)}
صف توسط دو باجه‌ی سنتی با نرخ سرویس دهی 600 نفر در ساعت سرویس‌دهی شود. مشابه قسمت قبل فرض کنید که زمان‌های سرویس‌دهی از توزیع نمایی پیروی کنند و استراتژی سرویس‌دهی FIFO است و پیدا کنید زمان انتظار مورد انتظار در حالت پایدار 
\lr{(Steady-state expected waiting time)}.

\section*{جواب سوال ۵}


برای محاسبه زمان انتظار مورد انتظار در هر دو سناریو، از فرمول‌های مربوط به سیستم‌های صف \(M/M/1\) و \(M/M/2\) استفاده می‌کنیم.

\textbf{الف)} برای یک باجه اتوماتیک با نرخ خدمت دهی \(1200\) نفر در ساعت، نرخ استفاده \(\rho\) برابر است با \(\frac{1000}{1200}\). زمان انتظار مورد انتظار در حالت پایدار محاسبه می‌شود به صورت:
\[
W = \frac{1}{\mu - \lambda} = \frac{1}{1200 - 1000} = 0.005 \text{ ساعت} \quad (\text{یا 18 ثانیه})
\]

\textbf{ب)} برای دو باجه سنتی با نرخ خدمت دهی \(600\) نفر در ساعت برای هر باجه، نرخ استفاده \(\rho\) برابر است با \(\frac{1000}{2 \times 600}\). با استفاده از فرمول‌های پیچیده‌تر برای سیستم \(M/M/2\)، زمان انتظار مورد انتظار به دست می‌آید:
\[
P0 = \left( \sum_{k=0}^{c-1} \frac{(c\rho)^k}{k!} + \frac{(c\rho)^c}{c! \times (1-\rho)} \right)^{-1}
\]
\[
W = \frac{P0 \times \rho}{c \times \mu \times (1 - \rho)^2} + \frac{1}{\mu} = 0.00394 \text{ ساعت} \quad (\text{یا 14.18 ثانیه})
\]

در نتیجه، از نظر زمان انتظار، دو باجه سنتی بهتر عمل می‌کنند و زمان انتظار کمتری نسبت به یک باجه اتوماتیک دارند. انتخاب بین این دو گزینه باید بر اساس عوامل دیگری مانند هزینه، راحتی استفاده و غیره انجام شود.