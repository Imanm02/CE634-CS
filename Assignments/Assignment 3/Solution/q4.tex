\section*{سوال ۴}
	
\textbf{سوال:}
	
فرض کنید یک سیستم صف
$M/M/1$
وجود دارد. 
	
\textbf{الف)} اثبات کنید که احتمال وجود \(n\) مشتری در مغازه برابر است با:
\[
P_n = (1 - \rho)\rho^n
\]
که در آن \(\rho = \frac{\lambda}{\mu}\) نرخ استفاده (\textit{utilization}) است.
	
\textbf{ب)} با استفاده از نتیجه بخش الف، اثبات کنید که تعداد مشتریان منتظر در صف به طور میانگین برابر است با:

\[
L_q = \frac{\rho^2}{1 - \rho}
\]

\section*{جواب سوال ۴}

\textbf{الف)} برای یک سیستم صف 
$M/M/1$
، فرمول استاندارد برای محاسبه احتمال وجود \(n\) مشتری در سیستم (هم در صف و هم در خدمت) استفاده می‌شود. این فرمول بر اساس این واقعیت است که احتمال ورود مشتریان به سیستم به صورت هندسی توزیع شده است. فرض کنید \(\rho = \frac{\lambda}{\mu}\) که \(\lambda\) نرخ ورود و \(\mu\) نرخ خدمت است. فرض می‌کنیم که سیستم در حالت تعادل است. پس داریم:
\[
P_n = (1 - \rho)\rho^n
\]
که در آن \(P_n\) احتمال وجود دقیقاً \(n\) مشتری در سیستم است. این فرمول از معادلات تعادل حالت پایا برای سیستم‌های صف 
$M/M/1$
به دست می‌آید. 

\textbf{ب)} برای محاسبه تعداد میانگین مشتریان منتظر در صف، از فرمول Little استفاده می‌کنیم. فرمول Little می‌گوید که \(L = \lambda W\)، که \(L\) تعداد میانگین مشتریان در سیستم و \(W\) میانگین زمان انتظار یک مشتری در سیستم است. برای یک سیستم 
$M/M/1$
، میانگین زمان انتظار در سیستم \(W\) برابر است با \(\frac{1}{\mu - \lambda}\). پس داریم:
\[
L = \lambda \times \frac{1}{\mu - \lambda} = \frac{\lambda}{\mu - \lambda} = \frac{\rho}{1 - \rho}
\]
از آنجا که \(L_q = L - \rho\) (که \(L_q\) تعداد میانگین مشتریان در صف است)، می‌توانیم بنویسیم:
\[
L_q = \frac{\rho}{1 - \rho} - \rho = \frac{\rho^2}{1 - \rho}
\]
که این فرمول نشان دهنده تعداد میانگین مشتریان منتظر در صف در یک سیستم صف 
$M/M/1$
است.
