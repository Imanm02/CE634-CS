\section*{سوال ۴}
	
\textbf{سوال:}
	
فرض کنید یک سیستم صف
$M/M/1$
وجود دارد. 
	
\textbf{الف)} اثبات کنید که احتمال وجود \(n\) مشتری در مغازه برابر است با:
\[
P_n = (1 - \rho)\rho^n
\]
که در آن \(\rho = \frac{\lambda}{\mu}\) نرخ استفاده (\textit{utilization}) است.
	
\textbf{ب)} با استفاده از نتیجه بخش الف، اثبات کنید که تعداد مشتریان منتظر در صف به طور میانگین برابر است با:

\[
L_q = \frac{\rho^2}{1 - \rho}
\]

\section*{جواب سوال ۴}

\textbf{الف)} در یک سیستم صف 
$M/M/1$
، احتمال وجود \(n\) مشتری در سیستم (هم در صف و هم در خدمت) با استفاده از فرمول استاندارد صف
$M/M/1$
محاسبه می‌شود:
\[
P_n = (1 - \rho)\rho^n
\]
که در آن \(\rho = \frac{\lambda}{\mu}\) است. این فرمول بر اساس این واقعیت است که احتمال ورود مشتریان به سیستم به صورت هندسی توزیع شده است.

\textbf{ب)} برای محاسبه تعداد میانگین مشتریان در صف، می‌توان از فرمول Little استفاده کرد. با توجه به اینکه \(L = \lambda W\) و \(L_q = L - \rho\) (که \(L\) تعداد میانگین مشتریان در سیستم و \(L_q\) تعداد میانگین مشتریان در صف است)، ما داریم:
\[
L_q = \lambda W - \rho
\]
از آنجایی که \(W = \frac{1}{\mu - \lambda}\) (زمان میانگین انتظار در سیستم)، می‌توان نوشت:
\[
L_q = \lambda \frac{1}{\mu - \lambda} - \rho = \frac{\rho}{1 - \rho} - \rho = \frac{\rho^2}{1 - \rho}
\]
که این نتیجه حاصل می‌شود.