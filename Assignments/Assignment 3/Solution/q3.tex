\section*{سوال ۳}

در این سوال قصد بررسی مدل صف در یک مطب را داریم. فرض کنید که در یک مطب دو پزشک وجود دارند که هرکدام از آنها در ۱۵ دقیقه، یک بیمار را معاینه و درمان می‌کنند و در هر یک ساعت، یک بیمار به مطب وارد می‌شود. فرض کنید که میدانیم این مطب امروز تنها ۱۰ بیمار خواهد داشت.

\begin{enumerate}
	\item تعداد میانگین افرادی که در مطب حضور دارند را بدست آورید.
	\item هر بیمار به طور میانگین چند دقیقه را در صف می‌گذارند؟
	\item هر پزشک به طور میانگین در چه نسبتی از یک ساعت، هیچ بیماری را ویزیت نمی‌کند؟
\end{enumerate}

\section*{جواب سوال ۳}

کد جواب این سوال نیز پیوست شده است.

میانگین تعداد افراد در مطب: از آنجایی که هر پزشک در هر ۱۵ دقیقه یک بیمار را معاینه می‌کند و هر ساعت یک بیمار وارد می‌شود، هر پزشک می‌تواند در هر ساعت ۴ بیمار را ویزیت کند. با دو پزشک موجود، آن‌ها می‌توانند در مجموع ۸ بیمار را در هر ساعت ویزیت کنند. با این حال، تنها یک بیمار در هر ساعت وارد می‌شود، بنابراین، در هر ساعت به طور میانگین \(\frac{1}{8}\) بیمار در مطب حضور خواهد داشت.

با استفاده از مدل صف
$M/M/c/ \infty / \infty$
، محاسبات زیر انجام شده‌اند:

\begin{itemize}
	\item
	میانگین تعداد افراد در مطب: حدود $0.254$ نفر.
	\item
	میانگین زمان انتظار هر بیمار در صف: حدود $15.24$ دقیقه.
	\item 
	میانگین زمان بیکاری هر پزشک: حدود $87.5$ درصد.
\end{itemize}