\section*{سوال ۱}



\section*{جواب سوال ۱}

برای محاسبه متوسط زمان تولید اولین نان در هر دو سناریو:

1. \( \mathbb{E}[T_1] \) (زمان برای تولید اولین نان، بدون توجه به نوع آن): متوسط زمان تولید اولین نان، خواه بربری یا سنگک، برابر است با \( \frac{1}{\lambda_b + \lambda_s} \). این به دلیل آن است که میزان وقوع رویداد در هر دو فرایند پواسون را می‌توان به صورت ترکیبی در نظر گرفت.

2. \( \mathbb{E}[T_2] \) (زمان برای تولید حداقل یک نان بربری و یک نان سنگک): متوسط زمان لازم برای تولید حداقل یک نان از هر نوع برابر است با \( \frac{\lambda_s}{\lambda_b(\lambda_b + \lambda_s)} \)، به شرطی که نرخ‌های \( \lambda_b \) و \( \lambda_s \) مثبت و متمایز باشند. این فرمول بر اساس خصوصیات توزیع‌های نمایی مستقل که مربوط به زمان‌های انتظار برای هر فرایند پواسون هستند، به دست می‌آید.

این محاسبات بر اساس اصول تئوری احتمال و فرایندهای پواسون انجام شده‌اند.
