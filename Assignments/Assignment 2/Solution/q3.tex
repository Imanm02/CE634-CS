\section*{سوال ۳}



\section*{جواب سوال ۳}

\textbf{الف) نشان دهید:} 
\[ \sum_{j=1}^{\infty} (a_1 + a_2 + \ldots + a_j) P(N = j) = \sum_{i=1}^{\infty} a_i P(N \geq i) \]
برای این بخش، از خاصیت انتظار متغیر تصادفی گسسته استفاده می‌کنیم. مجموع داخل پرانتز را به صورت تجمعی نویسیم و با تغییر ترتیب جمع‌زنی، این را به صورت زیر بیان می‌کنیم:
\[ = \sum_{i=1}^{\infty} a_i \sum_{j=i}^{\infty} P(N = j) = \sum_{i=1}^{\infty} a_i P(N \geq i) \]

\textbf{ب) ثابت کنید:} 
\[ E[N] = \sum_{j=1}^{\infty} P(N \geq j) \]
این یک نتیجه مستقیم از تعریف امید ریاضی است. امید ریاضی \( E[N] \) برابر است با مجموع تمام احتمالات که در آن \( N \) بزرگتر یا مساوی هر عدد طبیعی \( j \) است.

\textbf{ج) ثابت کنید:} 
\[ E[N(N + 1)] = 2 \sum_{j=1}^{\infty} j \cdot P(N \geq j) \]
برای این بخش، ما ابتدا \( N(N + 1) \) را به صورت \( N^2 + N \) بیان می‌کنیم و سپس از خاصیت امید ریاضی استفاده می‌کنیم. امید ریاضی \( E[N(N + 1)] \) را می‌توان به صورت مجموعه‌ای از احتمالات که در آن \( N \) بزرگتر یا مساوی هر عدد طبیعی \( j \) است، بیان کرد.