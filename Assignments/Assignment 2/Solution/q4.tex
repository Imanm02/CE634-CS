\section*{جواب سوال ۴}

\textbf{محاسبه امید ریاضی تعداد دفعاتی که باید تاس انداخته شود تا توالی عدد مضرب ۳ نباشد- عدد مضرب ۳ باشد - عدد مضرب ۳ نباشد مشاهده شود:}

برای حل این مسئله، ما از نظریه زنجیره‌های مارکوف استفاده می‌کنیم. سه حالت ممکن برای هر پرتاب تاس وجود دارد:
\begin{enumerate}
	\item عدد مضرب ۳ (حالت \(A\))
	\item عدد غیر مضرب ۳ و توالی شروع نشده (حالت \(B\))
	\item عدد غیر مضرب ۳ و توالی شروع شده (حالت \(C\))
\end{enumerate}

ماتریس انتقال \(P\) به صورت زیر است:
\[ P = \begin{bmatrix}
	\frac{1}{3} & \frac{2}{3} & 0 \\
	\frac{1}{3} & \frac{2}{3} & 0 \\
	0 & 0 & 1
\end{bmatrix} \]

برای محاسبه امید ریاضی تعداد دفعات پرتاب تاس تا رسیدن به حالت \(C\) بعد از حالت \(A\)، ما از خواص زنجیره‌های مارکوف استفاده می‌کنیم. این محاسبه شامل محاسبه زمان اولین ورود به حالت \(C\) از حالت \(A\) است.

\textbf{رسم زنجیره مارکوف مرتبط با این رویداد:}
برای رسم زنجیره مارکوف مرتبط با این رویداد، ما از یک نمودار حالت استفاده می‌کنیم که در آن انتقالات بین حالات \(A\)، \(B\)، و \(C\) نمایش داده می‌شود.
