\section*{سوال ۳}

در مورد فلوچارت موجود در اسلاید‌های ۲۸ و ۲۹ سری اول، به سوالات زیر پاسخ دهید:

\begin{enumerate}[الف]
	\item تفاوت دو مرحله‌ی \textit{Verification} و \textit{Validation} چیست؟
	\item در مرحله‌ی ۱۰، چه زمانی به تعداد اجرای بیشتری نیاز داریم؟ یک سناریو را ذکر کنید که در آن نیاز به تعداد اجراهای بیشتری داشته باشیم.
	\item آیا \textit{Model Conceptualization} اثری بر نوع داده‌هایی دارد که باید جمع‌آوری شود؟
	\item پس از انجام شبیه‌سازی در مرحله‌ی ۱۰، چه زمانی به مرحله‌ی ۸ 
	(\textit{Experimental Design}) می‌رویم 
	و چه زمانی به مرحله‌ی ۹
	(\textit{Production Runs and Analysis}) می‌رویم؟
\end{enumerate}

\section*{جواب سوال ۳}

\begin{enumerate}[label=\textbf{\arabic*}]
	\item \textbf{تفاوت \lr{Verification} و \lr{Validation}:} 
	\lr{Verification} به فرآیندی گفته می‌شود که در آن بررسی می‌کنیم که آیا مدل به درستی ساخته شده است یا خیر، یعنی «آیا ما مدل را به درستی ساخته‌ایم؟». این مرحله بر کیفیت و صحت فنی مدل تمرکز دارد و شامل تست‌هایی برای اطمینان از برنامه‌نویسی صحیح و رفع اشکالات احتمالی است. در مقابل، \lr{Validation} به سنجش اینکه آیا مدل ساخته شده واقعاً نماینده‌ی صحیحی از سیستم واقعی است می‌پردازد یعنی «آیا ما مدل را به درستی ساخته‌ایم؟». این مرحله شامل مقایسه نتایج مدل با داده‌های واقعی و تأیید صحت کلی مدل است.
	
	\item \textbf{نیاز به تعداد اجرای بیشتر:} 
	در مرحله‌ی ۱۰، نیاز به تعداد اجرای بیشتر معمولاً زمانی پیش می‌آید که واریانس نتایج شبیه‌سازی بسیار بالا باشد و بخواهیم اطمینان حاصل کنیم که نتایج ما قابل اعتماد هستند. 
	\newline
	\textit{سناریو:} اگر مدل شبیه‌سازی یک کسب‌وکار جدید است و می‌خواهیم اثر متغیرهای تصادفی مانند تقاضای مشتریان را در شرایط مختلف بررسی کنیم، ممکن است نیاز به اجرای متعدد شبیه‌سازی باشد تا بتوانیم یک تصویر دقیق از توزیع نتایج و ریسک‌های احتمالی به دست آوریم.
	
	\item \textbf{اثر \lr{Model Conceptualization} بر جمع‌آوری داده‌ها:} 
	قطعاً \lr{Model Conceptualization} اثری بر نوع داده‌هایی دارد که باید جمع‌آوری شود. این مرحله شامل تعریف مسئله و ساختار کلی مدل است و بر اساس آن، ما می‌توانیم مشخص کنیم که چه نوع داده‌هایی لازم است تا مدل بتواند سیستم واقعی را به درستی نمایش دهد.
	
	\item \textbf{مراحل بعدی پس از انجام شبیه‌سازی:} 
	پس از انجام شبیه‌سازی در مرحله‌ی ۱۰، اگر نتایج به دست آمده نیاز به بهبود داشته باشند یا اگر فرضیات مدل نیاز به تغییر داشته باشند، به مرحله‌ی ۸ (\lr{Experimental Design}) باز می‌گردیم. در این مرحله، ممکن است نیاز به تنظیم دوباره‌ی طرح آزمایشات و تغییر پارامترها یا فرضیات باشد. اگر نتایج مطلوب باشند و نیاز به بررسی‌های بیشتری نباشد، به مرحله‌ی ۹ (\textit{Production Runs and Analysis}) می‌رویم که در آن تعداد بیشتری از اجراها برای تأیید نتایج و تحلیل‌های نهایی انجام می‌شود.
\end{enumerate}