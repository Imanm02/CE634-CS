\section*{سوال ۴}

\section*{مرکز واکسن}

در این سوال، یک مرکز واکسن را بررسی می‌کنیم. هر روز تعدادی نفر به مرکز مراجعه می‌کنند. بسته به موجودی واکسن، ممکن است نوبت واکسن مراجعین به روزهای بعد منتقل شود.

این مرکز یک مسئول دارد که تنها هر ۴ روز یک بار انبار واکسن را بررسی می‌کند و واکسن سفارش می‌دهد.

تعداد واکسن‌های سفارش داده شده از فرمول زیر پیروی می‌کند:
\[ \text{\lr{ordered vaccines}} = 6 - \#(\text{\lr{remaining vaccines}}) + \#(\text{\lr{vaccine shortage}}) \]

که در آن، 
\lr{\#(vaccine shortage)}
تعداد کل مراجعینی است که واکسن خود را هنوز دریافت نکرده‌اند.

مثال: اگر در انتهای روز چهارم ۳ واکسن در انبار باقی مانده باشد و کمبودی نداشته باشیم، ۳ واکسن سفارش داده می‌شود. و اگر تنها ۲ واکسن کمبود داشته باشیم و انبار خالی باشد، ۸ واکسن سفارش داده می‌شود.


تعداد مراجعین (تقاضا) هر روز از جدول آبی پیروی می‌کند. برای بدست آوردن مقدار رندوم، از لیست ۱ مشابه اسلایدهای درس استفاده کنید.

هنگام سفارش واکسن، حداکثر تا ۳ روز طول می‌کشد تا واکسن‌ها وارد انبار شوند. زمان مورد نیاز برای افزایش ذخیره انبار از جدول قرمز پیروی می‌کند. برای اعداد رندوم، از لیست ۲ استفاده کنید.

شما باید سیستم را برای ۲۰ روز شبیه‌سازی کرده و جدول مربوطه را رسم کنید. فرض کنید در انتهای روز صفر، ۳ واکسن در انبار موجود بوده و کمبودی نداریم. همچنین این روز، روز بررسی انبار است. به عبارت دیگر، در پایان این روز، ۳ سفارش واکسن انجام می‌گیرد.

جدول شما باید شامل ستون‌های زیر باشد:

\begin{itemize}
	\item روز
	\item ذخیره انبار در ابتدای روز
	\item تعداد مراجعین روز
	\item ذخیره انبار در انتهای روز
	\item میزان کمبود در انتهای روز
	\item روز سفارش
	\item مقدار سفارش
	\item تعداد روزهای مانده تا افزایش ذخیره انبار
\end{itemize}

برای ستون‌هایی که میانگین معنادار دارد، آن را نمایش دهید. استفاده از دستورات اکسل در این مسئله مجاز نیست و جدول باید به صورت دستی پُر شود.

$$
List 1: 68, 32, 83, 6, 82, 21, 12, 32, 59, 48, 58, 12, 18, 48, 22, 57, 18, 89, 84, 51
$$

$$
List 2: 76, 30, 96, 37, 48, 70
$$

\begin{table}
	\centering
	\begin{tabular}{|c|c|}
		\arrayrulecolor{red}\hline
		\textbf{lead time} & \textbf{Probability} \\
		\arrayrulecolor{red}\hline
		$1$ & $0.5$ \\
		\hline
		$2$ & $0.4$ \\
		\hline
		$3$ & $0.1$ \\
		\arrayrulecolor{red}\hline
	\end{tabular}
\end{table}

\begin{table}
	\centering
	\begin{tabular}{|c|c|}
		\arrayrulecolor{blue}\hline
		\textbf{Demand} & \textbf{Probability} \\
		\hline
		$0$ & $0.1$ \\
		\hline
		$1$ & $0.25$ \\
		\hline
		$2$ & $0.45$ \\
		\hline
		$3$ & $0.2$ \\
		\hline
	\end{tabular}
\end{table}

\section*{جواب سوال ۴}

