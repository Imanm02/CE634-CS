\section*{سوال ۲}

\subsection*{الف}

درباره صحیح یا غلط بودن بودن جملات زیر استدلال کنید (در صورت غلط بودن برای آن مثال نقض بیاورید)

\begin{itemize}
	\item \textbf{۱-} مدل‌های شبیه‌سازی گسسته را فقط برای مدل کردن سیستم‌های گسسته می‌توان استفاده کرد.
	\item \textbf{۲-} مدل‌های شبیه‌سازی پیوسته را فقط برای مدل کردن سیستم‌های پیوسته می‌توان استفاده کرد.
\end{itemize}

\subsection*{ب}

طبق اسلایدها می‌دانیم هنگامی که محیط تاثیر اندکی روی سیستم دارد، می‌توانیم به سه صورت آن را در شبیه‌سازی استفاده کنیم. این روش‌ها را شرح داده و برای هر یک مثالی که استفاده از آن روش مطلوب است را بیان کنید.

\section*{جواب سوال ۲}

\subsection*{الف}

\begin{itemize}
	\item \textbf{۱-} مدل‌های شبیه‌سازی گسسته را فقط برای مدل کردن سیستم‌های گسسته می‌توان استفاده کرد.
	\begin{itemize}
		\item \textbf{پاسخ:} این جمله \textbf{غلط} است. گاهی اوقات برای ساده‌سازی مدل‌سازی یک سیستم پیوسته، می‌توانیم از مدل‌های گسسته استفاده کنیم. مثلاً در مدل‌سازی جریان ترافیک که به طور واقعی پیوسته است، می‌توانیم از مدل‌های گسسته برای شبیه‌سازی حرکت خودروها در بازه‌های زمانی معین استفاده کنیم.
	\end{itemize}
	\item \textbf{۲-} مدل‌های شبیه‌سازی پیوسته را فقط برای مدل کردن سیستم‌های پیوسته می‌توان استفاده کرد.
	\begin{itemize}
		\item \textbf{پاسخ:} این جمله نیز \textbf{غلط} است. در برخی موارد، برای به دست آوردن نتایج دقیق‌تر و همچنین برای مدل‌سازی تغییرات گسسته با دقت بالا، می‌توان از مدل‌های شبیه‌سازی پیوسته استفاده کرد. به عنوان مثال، می‌توانیم سیستم صف گسسته را با استفاده از مدل پیوسته شبیه‌سازی کنیم تا نحوه تغییر تراکم صف در طول زمان را ببینیم.
	\end{itemize}
\end{itemize}

\subsection*{ب}

هنگامی که محیط تاثیر اندکی روی سیستم دارد، سه رویکرد برای مدل‌سازی و شبیه‌سازی وجود دارد:

\begin{enumerate}
	\item \textbf{در نظر گرفتن موارد خارجی به عنوان ورودی‌ها:}
	در این روش، فرض بر این است که موارد خارجی می‌توانند به صورت ورودی‌های کنترل‌شده‌ای در مدل گنجانده شوند که تاثیر آن‌ها قابل پیش‌بینی و مدیریت است.
	\newline
	\textit{مثال:} در شبیه‌سازی پخش یک بیماری، تعداد افراد وارد شونده به یک شهر می‌تواند به عنوان ورودی مدل در نظر گرفته شود که بر میزان گسترش بیماری تاثیر می‌گذارد.
	
	\item \textbf{گسترش تعریف سیستم برای شامل کردن عوامل خارجی:}
	این رویکرد شامل توسعه حدود سیستم برای گنجاندن عوامل خارجی به عنوان بخشی از خود سیستم است.
	\newline
	\textit{مثال:} در مدل‌سازی یک کسب‌وکار، عوامل محیطی مانند رقابت و تقاضای بازار می‌توانند به عنوان بخشی از سیستم کسب‌وکار تعریف شوند تا تاثیرات آن‌ها به صورت دقیق‌تری مورد بررسی قرار گیرد.
	
	\item \textbf{نادیده گرفتن عوامل خارجی:}
	در صورتی که تاثیر محیط بر سیستم بسیار ناچیز باشد، می‌توان آن‌ها را کاملاً نادیده گرفت.
	\newline
	\textit{مثال:} برای مدل‌سازی فرایند داخلی یک دستگاه تولیدی که در محیط کنترل‌شده قرار دارد، ممکن است تاثیرات محیطی نظیر دما و رطوبت خارجی قابل صرف نظر باشند.
\end{enumerate}