\section*{سوال ۵}

در این سوال، یک مرکز بازسازی کالا را شبیه‌سازی می‌کنیم. کالاهای معیوبی که توسط مشتریان به محل‌های مشخص مانند خرده فروشان بازگشت داده می‌شوند، جمع‌آوری شده و به این مرکز ارسال می‌شوند. فاصله زمانی بین ورود کالاها به مرکز، از ستون 
\lr{"Interarrival Time"}
در جدول ۱ به دست می‌آید. برای تولید اعداد تصادفی مورد نیاز در به دست آوردن زمان‌ها می‌توانید از روش دلخواه خود استفاده کنید.

هر کالا پس از ورود به ایستگاهی در این مرکز مورد بررسی قرار می‌گیرد تا عیوب آن شناسایی شده و نیازهای آن جهت تعمیر مشخص گردد. زمان مورد نیاز برای هر کالا در ایستگاه، مقداری تصادفی می‌باشد. ایستگاه نخست نام "Able" و ایستگاه دوم نام "Baker" دارد. برای به دست آوردن زمان مورد نیاز بازسازی هر کالا از ستون‌های هر ایستگاه در جدول ۱ استفاده کنید.

از آنجا که زمان بازسازی کالاها نسبتا طولانی است و هر ایستگاه در یک لحظه تنها می‌تواند یک کالا را بازسازی کند، یکی از مشکلات، طولانی شدن زمان انتظار هر کالا در صف و همچنین زمان کل لازم برای بازسازی کالاها می‌باشد. بدین منظور، قصد داریم سه حالت را برای ساختار صف‌های مرکز و همچنین پالیسی ورود کالاها به صف‌ها بررسی کنیم.

برای هر یک از حالات زیر، ۲۰ کالای ورودی را با رسم جدول، شبیه‌سازی کرده و میانگین مدت زمان انتظار در صف و مدت زمان کل شبیه‌سازی را با یکدیگر مقایسه کنید. دقت کنید که هر ایستگاه تنها به یک کالا در هر زمان سرویس می‌دهد. همچنین محدودیتی برای طول صف نداریم.

حالت ۱: هر یک از ایستگاه‌ها صف انتظار مخصوص خود را دارد. کالاها هنگام ورود به مرکز، وارد صف ایستگاهی می‌شوند که مدت زمان کمتری برای بازسازی آن لازم دارد (از زمان شناسایی شدن عیوب صرف‌نظر کنید).

حالت ۲: مشابه حالت ۱، هر ایستگاه صف انتظار مخصوص خود را دارد. در این حالت، کالاها هنگام ورود به مرکز، وارد صف کوتاه‌تر می‌شوند. در صورت تساوی طول صف‌ها، صف مشابه حالت ۱ (ایستگاه با زمان بازسازی کم‌تر) انتخاب می‌گردد.

حالت ۳: در این حالت تنها یک صف انتظار کلی وجود دارد. همچنین مشابه حالت ۱، هر کالا توسط ایستگاهی که زمان کمتری نیاز دارد، بازسازی می‌شود. هر کالا پس از رسیدن به ابتدای صف، اگر ایستگاه مطلوبش خالی بود، وارد ایستگاه شده و یکی جلو می‌رود، وگرنه در ابتدای صف منتظر می‌ماند تا ایستگاه خالی شود (دقت کنید در این حالت، تمامی کالاها در صف منتظر می‌مانند).

نکته: در هر حالت، در صورت تساوی زمان بازسازی یا طول صف، ایستگاه "Able" انتخاب می‌گردد.

در این سوال مجاز به استفاده از اکسل هستید.

\begin{table}[h]
	\centering
	\small % Smaller font size; you can also use \scriptsize or \footnotesize
	\setlength{\tabcolsep}{3pt} % Reduce the space between columns
	\renewcommand{\arraystretch}{0.8} % Reduce the space between rows
	\begin{tabular}{|c|c|c|c|c|c|}
		\arrayrulecolor{red}\hline
		\textbf{\lr{Probability}} & \textbf{\lr{Baker Service Time}} & \textbf{\lr{Probability}} & \textbf{\lr{Service Time}} & \textbf{\lr{Probability}} & \textbf{\lr{Interval Time}} \\
		\hline
		$0.38$ & $5$ & $0.32$ & $4$ & $0.2$ & $1$ \\
		\hline
		$0.26$ & $6$ & $0.26$ & $5$ & $0.45$ & $2$ \\
		\hline
		$0.19$ & $7$ & $0.24$ & $6$ & $0.2$ & $3$ \\
		\hline
		$0.17$ & $8$ & $0.18$ & $8$ & $0.15$ & $4$ \\
		\hline
	\end{tabular}
\end{table}

\section*{جواب سوال ۵}
