\section*{جواب سوال ۲}

این تسک مربوط به صفحه‌ی ۳۲ اسلاید ۴ می‌باشد.

برای پیدا کردن بردار احتمال در حالت پایدار (steady-state) پس از \(n \to \infty\) گام برای یک زنجیره مارکوف، هدف ما یافتن احتمالات پایدار است. بردار احتمال پایدار، \(\mathbf{p}_\infty\)، معادله \(\mathbf{p}_\infty = \mathbf{P} \mathbf{p}_\infty\) را ارضا می‌کند، که در آن \(\mathbf{P}\) ماتریس انتقال زنجیره مارکوف است. این شرط نشان می‌دهد که توزیع احتمالات پس از انتقال‌های بیشتر تغییر نمی‌کند و رفتار بلند مدت سیستم را نشان می‌دهد.

با توجه به بردارهای \(\mathbf{p}_3\)، \(\mathbf{p}_4\)، و \(\mathbf{p}_{10}\) که داریم، واضح است که احتمالات به یک بردار خاص همگرا می‌شوند وقتی که \(n\) افزایش می‌یابد. این بردار همگرایی همان بردار احتمال پایداری است که ما به دنبالش هستیم برای \(n \to \infty\).

بدون داشتن ماتریس انتقال \(\mathbf{P}\) به صورت صریح، من روش کلی یافتن \(\mathbf{p}_\infty\) را نشان می‌دهم با فرض داشتن یک بردار حالت اولیه \(\mathbf{p}_0\). این روش شامل حل کردن سیستم معادلات خطی است که توسط \(\mathbf{p}_\infty = \mathbf{P} \mathbf{p}_\infty\) به علاوه شرطی که جمع احتمالات در \(\mathbf{p}_\infty\) برابر با 1 باشد (چون کل احتمال باید حفظ شود)، تشکیل شده است.

از این نمونه مشاهده می‌شود که بردارها به سمت مقادیر خاصی همگرا می‌شوند. مقادیر ارائه شده برای \(\mathbf{p}_{3}\)، \(\mathbf{p}_{4}\)، و \(\mathbf{p}_{10}\) نشان می‌دهد که بردار حالت پایدار \(\mathbf{p}_\infty\) تقریباً \([0.330357, 0.303571, 0.366072]\) است. این نشان می‌دهد که فارغ از بردار حالت اولیه (\([0, 1, 0]\) یا \([0, 0, 1]\))، سیستم پس از گذشت زمان کافی به این حالت پایدار همگرا می‌شود.

برای محاسبه آن به صورت رسمی برای هر بردار حالت اولیه مانند \(\mathbf{p}_0 = [0, 1, 0]\) یا \(\mathbf{p}_0 = [0, 0, 1]\)، با فرض داشتن یک ماتریس انتقال \(\mathbf{P}\) مشخص، ما باید سیستم زیر را حل کنیم:

\begin{itemize}
	\item معادله \(\mathbf{p}_\infty = \mathbf{P} \mathbf{p}_\infty\) را تنظیم کنید.
	\item شرط نرمال‌سازی، \(\sum p_{\infty,i} = 1\) را اضافه کنید.
	\item سیستم را برای \(\mathbf{p}_\infty\) حل کنید.
\end{itemize}

\textbf{تحلیل}

\begin{itemize}
	\item \textbf{همگرایی}: هر دو بردار اولیه \([0, 1, 0]\) و \([0, 0, 1]\) به همان بردار حالت پایدار همگرا می‌شوند، که نشان‌دهنده رفتار بلندمدت سیستم است که مستقل از حالت اولیه آن است.
	\item \textbf{تفسیر}: بردار حالت پایدار \([0.330357, 0.303571, 0.366072]\) نشان می‌دهد که در طولانی مدت، سیستم به گونه‌ای ثابت می‌شود که توزیع احتمال بین حالات با انتقال‌های بیشتر تغییر نمی‌کند. مقادیر دقیق نشان‌دهنده احتمال بلندمدت بودن سیستم در هر یک از حالات مربوطه است.
\end{itemize}
