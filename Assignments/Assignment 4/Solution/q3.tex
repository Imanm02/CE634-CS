\section*{سوال ۳}

در این سوال قصد بررسی مدل صف در یک مطب را داریم. فرض کنید که در یک مطب دو پزشک وجود دارند که هرکدام از آنها در ۱۵ دقیقه، یک بیمار را معاینه و درمان می‌کنند و در هر یک ساعت، یک بیمار به مطب وارد می‌شود. فرض کنید که میدانیم این مطب امروز تنها ۱۰ بیمار خواهد داشت.

\begin{enumerate}
	\item تعداد میانگین افرادی که در مطب حضور دارند را بدست آورید.
	\item هر بیمار به طور میانگین چند دقیقه را در صف می‌گذارند؟
	\item هر پزشک به طور میانگین در چه نسبتی از یک ساعت، هیچ بیماری را ویزیت نمی‌کند؟
\end{enumerate}

\section*{جواب سوال ۳}

میانگین تعداد افراد در مطب: از آنجایی که هر پزشک در هر ۱۵ دقیقه یک بیمار را معاینه می‌کند و هر ساعت یک بیمار وارد می‌شود، هر پزشک می‌تواند در هر ساعت ۴ بیمار را ویزیت کند. از آنجا که دو پزشک وجود دارد، آن‌ها می‌توانند در مجموع در هر ساعت ۸ بیمار را ویزیت کنند. اما تنها یک بیمار در هر ساعت وارد می‌شود. بنابراین، در هر ساعت به طور میانگین \(\frac{1}{8}\) بیمار در مطب حضور خواهد داشت.

میانگین زمان انتظار هر بیمار در صف: از آنجا که پزشکان قادر به ویزیت ۸ بیمار در ساعت هستند و فقط یک بیمار در هر ساعت وارد می‌شود، بنابراین به ندرت صفی ایجاد می‌شود. در نتیجه، میانگین زمان انتظار در صف بسیار کم خواهد بود و می‌توان آن را صفر در نظر گرفت.

میانگین زمان بیکاری هر پزشک: از آنجا که هر پزشک می‌تواند ۴ بیمار را در یک ساعت ویزیت کند، اما تنها یک بیمار در هر ساعت وارد می‌شود، پزشکان بیشتر وقت خود را بدون ویزیت بیمار سپری می‌کنند. بنابراین، هر پزشک به طور میانگین \(\frac{3}{4}\) یا ۷۵٪ از زمان یک ساعت بیکار خواهد بود.
