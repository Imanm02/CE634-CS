\section*{سوال ۶}

در این بخش قصد پیاده سازی الگوریتم های تولید اعداد شبه تصادفی و روش های تست آنها را داریم.

\textbf{بخش اول: پیاده سازی الگوریتم های تولید اعداد شبه تصادفی:}
\begin{itemize}
	\item در ابتدا یک تابع تولید عدد با الگوریتم \lr{CLCG} را پیاده کنید و بعد با استفاده از این تابع، الگوریتم \lr{LCM} را به صورت یک تابع پیاده کنید. در انتخاب پارامتر های الگوریتم ها آزاد هستید.
	\item در آخر، به عنوان یک روش متفاوت دیگر تولید اعداد شبه تصادفی، الگوریتم \lr{xor-shift} را مشابه بخش قبلی، پیاده کنید. (همچنین در مستند تمرین این بخش، توضیحی در مورد این الگوریتم ارائه دهید.)
\end{itemize}

\textbf{بخش دوم: تولید اعداد}
\begin{itemize}
	\item به ازای هرکدام از روش های پیاده سازی شده در بخش قبل، ۱۰۰۰ عدد ایجاد کنید و نمودار هیستوگرام هرکدام از آنها را رسم کنید.
\end{itemize}

\textbf{بخش سوم: تست اعداد}
\begin{itemize}
	\item در این بخش باید سه روش \lr{KS-test}، \lr{chi-square test} و \lr{auto-correlation test} را پیاده کنید.
	\begin{itemize}
		\item \lr{KS-test}: این تست را به صورت یک تابع ایجاد کنید که با ورودی گرفتن اعداد تولید شده و مقدار \lr{level of significance} نتیجه تست را چاپ کند.
		\item \lr{Chi-square test}: این تست نیز مشابه بخش قبل پیاده شود.
		\item \lr{Auto-correlation test}: این تست در قالب یک تابع ایجاد شود که با ورودی گرفتن اعداد تولید شده به همراه \lr{level of significance} و اندیس اولین عدد در تست و مقدار \lr{lag}، نتیجه تست را ایجاد کند.
	\end{itemize}
	\item برای این بخش تمرین می توانید از یکی از زبان های برنامه نویسی پایتون، جاوا، سی یا سی پلاس پلاس استفاده کنید. برای ارسال پاسخ این بخش، به همراه کد پیاده سازی باید یک مستند حاوی توضیحات لازم برای بخش های مختلف کد را ارائه دهید.
	\item نکته: برای بدست آوردن مقادیر بحرانی یا \lr{p-value} در تست های مختلف، میتوانید از کتابخانه های در دسترس در زبان برنامه نویسی مد نظر استفاده کنید و در پیاده سازی شما این مقادیر نباید \lr{hard code} شوند.
\end{itemize}

\section*{جواب سوال ۶}

\textbf{توضیح الگوریتم \lr{xor-shift} :}

الگوریتم \lr{xor-shift} یکی از روش‌های تولید کننده اعداد شبه تصادفی است که بر پایه عملیات‌های بیتی پایه‌ای نظیر XOR (تفاوت منطقی) و Shift (جابجایی بیتی) عمل می‌کند. این الگوریتم از یک عدد اولیه (seed) شروع کرده و با انجام یک سری عملیات‌های xor و shift بر روی این عدد، عدد جدیدی را تولید می‌کند. 

فرایند کاری الگوریتم به این صورت است:
\begin{itemize}
	\item ابتدا عدد اولیه (seed) با استفاده از عملیات shift به چپ و سپس عملیات xor با خودش ترکیب می‌شود.
	\item سپس عدد حاصل از مرحله قبلی با استفاده از عملیات shift به راست و دوباره عملیات xor با خودش ترکیب می‌شود.
	\item در آخر، عدد حاصل از مرحله دوم دوباره با استفاده از عملیات shift به چپ و عملیات xor با خودش ترکیب می‌شود تا عدد شبه تصادفی جدیدی به دست آید.
\end{itemize}

این الگوریتم به دلیل سادگی و کارایی بالا در تولید اعداد شبه تصادفی محبوب است و در بسیاری از کاربردها استفاده می‌شود.