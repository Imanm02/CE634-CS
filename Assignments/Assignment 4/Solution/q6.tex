\section*{سوال ۶}

برای تجزیه و تحلیل داده‌های موجود در دو فایل نمونه، ابتدا نمودار Q-Q را برای داده‌های فایل اول با توزیع ویبول با مقیاس \( \text{scale} = 2 \) و شکل \( \text{shape} = 1.6 \) ترسیم کردیم. در این نمودار، نقاط باید در امتداد خط قرمز قرار بگیرند اگر داده‌ها از توزیع ویبول پیروی کنند. مشاهده شد که داده‌ها تا حدودی با این توزیع همخوانی دارند، اما برخی انحرافات وجود دارد، به‌ویژه در دو انتهای نمودار.

سپس نمودار Q-Q برای همین داده‌ها را با توزیع نرمال با میانگین \( \mu = 0 \) و واریانس \( \sigma^2 = 2 \) ترسیم کردیم. در این نمودار، نقاط باید در امتداد خط قرمز قرار بگیرند اگر داده‌ها از توزیع نرمال پیروی کنند. مشاهده شد که داده‌ها با توزیع نرمال کمتر همخوانی دارند نسبت به توزیع ویبول، به ویژه در انتهاهای نمودار.

در نهایت، برای بررسی اینکه آیا داده‌های هر دو فایل ممکن است از یک توزیع مشترک پیروی کنند یا خیر، نمودار Q-Q را با استفاده از داده‌های فایل اول به عنوان \( X \) و داده‌های فایل دوم به عنوان \( Y \) ترسیم کردیم. در این نمودار، مقایسه دو مجموعه داده با یکدیگر صورت گرفته است. مشاهده می‌شود که توزیع داده‌ها در هر دو فایل تفاوت‌هایی دارند، اما روند کلی آن‌ها نسبتاً مشابه است. این نشان می‌دهد که ممکن است هر دو مجموعه داده از یک توزیع مشترک پیروی کنند، هرچند که این توزیع مشترک ممکن است دقیقاً توزیع نرمال نباشد.

به طور خلاصه، بر اساس تجزیه و تحلیل‌های نمودار Q-Q ، داده‌های فایل اول به نظر می‌رسد که بهتر با توزیع ویبول همخوانی دارند تا توزیع نرمال، و داده‌های هر دو فایل ممکن است از یک توزیع مشترک پیروی کنند که این توزیع مشترک لزوماً توزیع نرمال نیست.

\section*{جواب سوال ۶}

برای تحلیل داده‌های موجود در دو فایل نمونه، ابتدا نمودار Q-Q را برای داده‌های فایل اول با توزیع ویبول با مقیاس \( \text{scale} = 2 \) و شکل \( \text{shape} = 1.6 \) ترسیم کردیم. در این نمودار، نقاط باید در امتداد خط قرمز قرار بگیرند اگر داده‌ها از توزیع ویبول پیروی کنند. مشاهده شد که داده‌ها تا حدودی با این توزیع همخوانی دارند، اما برخی انحرافات وجود دارد، به‌ویژه در دو انتهای نمودار.

سپس نمودار Q-Q برای همین داده‌ها را با توزیع نرمال با میانگین \( \mu = 0 \) و واریانس \( \sigma^2 = 2 \) ترسیم کردیم. در این نمودار، نقاط باید در امتداد خط قرمز قرار بگیرند اگر داده‌ها از توزیع نرمال پیروی کنند. مشاهده شد که داده‌ها با توزیع نرمال کمتر همخوانی دارند نسبت به توزیع ویبول، به ویژه در انتهاهای نمودار.

در نهایت، برای بررسی اینکه آیا داده‌های هر دو فایل ممکن است از یک توزیع مشترک پیروی کنند یا خیر، نمودار Q-Q را با استفاده از داده‌های فایل اول به عنوان \( X \) و داده‌های فایل دوم به عنوان \( Y \) ترسیم کردیم. در این نمودار، مقایسه دو مجموعه داده با یکدیگر صورت گرفته است. مشاهده می‌شود که توزیع داده‌ها در هر دو فایل تفاوت‌هایی دارند، اما روند کلی آن‌ها نسبتاً مشابه است. این نشان می‌دهد که ممکن است هر دو مجموعه داده از یک توزیع مشترک پیروی کنند، هرچند که این توزیع مشترک ممکن است دقیقاً توزیع نرمال نباشد.

به طور خلاصه، بر اساس تجزیه و تحلیل‌های نمودار Q-Q، داده‌های فایل اول به نظر می‌رسد که بهتر با توزیع وایبول همخوانی دارند تا توزیع نرمال، و داده‌های هر دو فایل ممکن است از یک توزیع مشترک پیروی کنند که این توزیع مشترک لزوماً توزیع نرمال نیست. ولی برای نتیجه‌گیری بهتر است بگوییم توزیع آن‌ها یکی نیست چون تفاوت در نمودار آخر آشکار است و بعید است توزیع دو نوع داده یکسان باشد.