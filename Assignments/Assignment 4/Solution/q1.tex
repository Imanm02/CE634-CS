\section*{سوال ۱}

۱. به سوالات زیر پاسخ دهید:

\begin{enumerate}
	\item مدل‌های صف با جمعیت نامحدود \lr{(infinite population)} و جمعیت محدود \lr{(finite population)} را توصیف کنید و تفاوت‌های آن‌ها را بیان نمایید.
	\item انواع نظم‌دهی در صف‌ها یا نظام صف‌بندی \lr{(Queue discipline)} چیست؟ سه مورد از این انواع را نام برده و هر کدام را به طور خلاصه شرح دهید.
	\item جریان‌های شماره‌گذاری تصادفی \lr{(random-number streams)} را توضیح دهید.
\end{enumerate}

\section*{جواب سوال ۱}

\begin{enumerate}
	\item \textbf{مدل‌های صف با جمعیت نامحدود و جمعیت محدود:}
	جمعیت نامحدود به معنای آن است که تعداد مشتریان بالقوه برای ورود به صف بی‌نهایت است، بدین معنی که همیشه مشتری جدیدی برای ورود به صف وجود دارد و محدودیتی برای تعداد کل مشتریان وجود ندارد. این مدل معمولاً در سیستم‌هایی با ترافیک بالا مانند وب‌سایت‌ها یا سیستم‌های تلفنی مورد استفاده قرار می‌گیرد. در مقابل، جمعیت محدود به وضعیتی اشاره دارد که تعداد مشتریانی که ممکن است وارد صف شوند محدود است. این مدل در موقعیت‌هایی که تعداد کاربران سیستم محدود است، مثلاً در یک کسب‌وکار کوچک یا سیستمی با تعداد محدود کاربر، کاربرد دارد.
	
	\item \textbf{انواع نظم دهی در صف‌ها:}
	\begin{itemize}
		\item \textit{\lr{FIFO (First In, First Out)}}: این رویه بر اساس ترتیب ورود مشتریان به صف عمل می‌کند. اولین مشتری که وارد صف می‌شود، اولین کسی است که خدمات دریافت می‌کند. این روش در اکثر موقعیت‌های رایج مانند صف‌های بانک یا فروشگاه‌ها به کار می‌رود.
		\item \textit{\lr{LIFO (Last In, First Out)}}: در این روش، آخرین مشتری که وارد صف می‌شود، اولین کسی است که خدمات دریافت می‌کند. این روش عمدتاً در موقعیت‌های خاص مانند برخی فرآیندهای صنعتی یا در مدیریت داده‌ها به کار می‌رود.
		\item \textit{اولویت‌بندی}: در این روش، مشتریان بر اساس اولویت‌های مختلف خدمات دریافت می‌کنند. اولویت‌ها می‌توانند بر اساس فاکتورهای مختلفی مانند اورژانسی بودن، وضعیت ویژه یا اهمیت مشتری تعیین شوند. این روش در موقعیت‌هایی مانند بیمارستان‌ها یا مراکز خدماتی که نیاز به سرویس‌دهی فوری دارند، کاربرد دارد.
	\end{itemize}
	
	\item \textbf{جریان‌های شماره‌گذاری تصادفی:}
	جریان‌های شماره‌گذاری تصادفی ابزارهایی هستند که برای تولید دنباله‌هایی از اعداد تصادفی در شبیه‌سازی‌ها استفاده می‌شوند. این اعداد به منظور مدل‌سازی رفتارهای تصادفی در سیستم‌های واقعی استفاده می‌شوند، مانند تعیین زمان‌های ورود تصادفی مشتریان به صف یا زمان‌های متفاوت سرویس‌دهی در یک مرکز خدماتی. استفاده از این جریان‌ها امکان پذیرش یک مدل شبیه‌سازی را که به‌خوبی پویایی‌های واقعی را بازتاب می‌دهد، فراهم می‌کند.
\end{enumerate}