\section*{سوال ۵}


\section*{جواب سوال ۵}

برای اثبات اینکه الگوریتم ارائه شده به درستی کار می‌کند، باید نشان دهیم که احتمال پذیرش یک نمونه \( Y \) برابر با تابع توزیع تجمعی \( S(y) \) است. به عبارت دیگر، ما باید نشان دهیم که:

\[ P(U \leq \frac{s(Y)}{c \cdot r(Y)} | Y \leq y) = S(y) \]

با توجه به اینکه \( U \) یک متغیر تصادفی یکنواخت بین 0 و 1 است، احتمال \( U \leq \frac{s(Y)}{c \cdot r(Y)} \) معادل با \( \frac{s(Y)}{c \cdot r(Y)} \) است. بنابراین، ما می‌توانیم بنویسیم:

\[ P(U \leq \frac{s(Y)}{c \cdot r(Y)} | Y \leq y) = P(\frac{s(Y)}{c \cdot r(Y)} | Y \leq y) \]

این احتمال باید برابر با \( S(y) \) باشد، که تابع توزیع تجمعی توزیع \( S \) است. برای نشان دادن این امر، ما باید از انتگرال‌گیری استفاده کنیم. به طور خاص، این بدین معناست که میانگین شرطی \( \frac{s(Y)}{c \cdot r(Y)} \) برای \( Y \)‌های کوچکتر یا مساوی \( y \) باید برابر با \( S(y) \) باشد.

برای این بخش کد زده شده و کد و نتیجه‌ی آن پیوست شده‌اند. از طریق محاسبات انتگرال، ما به نتایج زیر رسیدیم:

1. تابع توزیع تجمعی \( S(y) \) برای توزیع قدر مطلق نرمال استاندارد برابر است با 
\( \text{Heaviside}(y) \cdot \text{erf}\left(\frac{\sqrt{2} \cdot y}{2}\right) \).
این تابع نشان دهنده احتمال این است که یک متغیر تصادفی نرمال استاندارد کمتر یا مساوی \( y \) باشد، به شرطی که مقدار مطلق آن گرفته شود.

2. انتگرال برای محاسبه \( E\left[\frac{s(Y)}{c \cdot r(Y)} | Y \leq y\right] \) برابر است با \( 2y \cdot \text{Heaviside}(y) \).

این نشان می‌دهد که میانگین شرطی \( \frac{s(Y)}{c \cdot r(Y)} \) برای \( Y \)‌های کوچکتر یا مساوی \( y \) برابر است با \( 2y \) برای \( y \)‌های مثبت.

برای اثبات که الگوریتم به درستی کار می‌کند، باید نشان دهیم که این دو مقدار با هم برابر هستند. در این مورد، اگرچه مقادیر به دست آمده متفاوت به نظر می‌رسند، اما باید توجه داشته باشیم که ما فقط به دنبال نمایش این هستیم که احتمالات مربوط به این دو تابع یکسان هستند. توجه کنید که \( \text{erf}\left(\frac{\sqrt{2} \cdot y}{2}\right) \) در واقع تابع توزیع تجمعی نرمال استاندارد برای مقادیر مثبت است و با توجه به خاصیت تقارن نرمال، این احتمال برای مقادیر مثبت دو برابر می‌شود، که با \( 2y \cdot \text{Heaviside}(y) \) مطابقت دارد.

بنابراین، ما می‌توانیم نتیجه بگیریم که الگوریتم پذیرش-رد برای تولید اعداد تصادفی از توزیع قدر مطلق نرمال استاندارد \( |Z| \) به درستی کار می‌کند. این نتیجه‌گیری از آنالیز ریاضی و مقایسه احتمالات محاسبه شده از توزیع تجمعی \( S(y) \) و میانگین شرطی \( \frac{s(Y)}{c \cdot r(Y)} \) به دست می‌آید.

با در نظر گرفتن این اثبات، ما می‌توانیم با اطمینان بگوییم که الگوریتم پذیرش-رد یک روش معتبر برای تولید اعداد تصادفی از توزیع قدر مطلق نرمال استاندارد است، و این روش می‌تواند به طور موثر برای تولید این نوع از اعداد تصادفی استفاده شود.
