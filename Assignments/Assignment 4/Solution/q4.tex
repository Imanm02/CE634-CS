\section*{سوال ۴}
	
داده‌های جدول زیر، نمونه‌های جمع‌آوری شده از مدت زمان سرویس در یک سیستم صف هستند. با کمک این داده‌ها، یک جدول برای تولید زمان‌های سرویس دهی ایجاد کنید (مشابه اسلاید ۱۶ از فصل ۷) و برای ۵ عدد تصادفی \( R_i \) زمان سرویس متناظر را تعیین کنید. برای تولید \( R_i \) می‌توانید از روشی دلخواه استفاده کنید.

\begin{center}
	\begin{tabular}{ |c|c| }
		\hline
		Interval (seconds) & Frequency \\
		\hline
		15-30 & 10 \\
		30-45 & 20 \\
		45-60 & 25 \\
		60-90 & 35 \\
		90-120 & 30 \\
		120-180 & 20 \\
		180-300 & 10 \\
		\hline
	\end{tabular}
\end{center}

\section*{جواب سوال ۴}

جدول تولید زمان‌های سرویس دهی بر اساس داده‌های جدول موجود:

\begin{center}
	\begin{tabular}{ |c|c|c|c| }
		\hline
		Interval (seconds) & Frequency & Relative Frequency & Cumulative Distribution \\
		\hline
		$15-30$ & $10$ & $0.067$ & $0.067$ \\
		$30-45$ & $20$ & $0.133$ & $0.200$ \\
		$45-60$ & $25$ & $0.167$ & $0.367$ \\
		$60-90$ & $35$ & $0.233$ & $0.600$ \\
		$90-120$ & $30$ & $0.200$ & $0.800$ \\
		$120-180$ & $20$ & $0.133$ & $0.933$ \\
		$180-300$ & $10$ & $0.067$ & $1.000$ \\
		\hline
	\end{tabular}
\end{center}

برای پنج عدد تصادفی \( R_i \) تولید شده، بازه‌های زمانی متناظر به شرح زیر هستند:

\begin{itemize}
	\item \( R_1 = 0.333 \): بازه زمانی 45-60
	\item \( R_2 = 0.686 \): بازه زمانی 90-120
	\item \( R_3 = 0.244 \): بازه زمانی 45-60
	\item \( R_4 = 0.430 \): بازه زمانی 60-90
	\item \( R_5 = 0.486 \): بازه زمانی 60-90
\end{itemize}

در این کد، ابتدا داده‌های مربوط به زمان سرویس در یک سیستم صف و فراوانی هر بازه زمانی را در یک دیکشنری به نام data ذخیره می‌کنیم. سپس، این دیکشنری را به یک DataFrame پانداس تبدیل می‌کنیم تا بتوانیم روی داده‌ها به راحتی عملیات‌های مختلفی انجام دهیم.

توضیح خطوط کد:

ایجاد
\lr{DataFrame}
از داده‌ها:

\lr{df = pd.DataFrame(data)}:
این خط کد، داده‌های موجود در دیکشنری
\lr{data}
را به یک
\lr{DataFrame}
پانداس تبدیل می‌کند.

محاسبه فراوانی کل:

\lr{total_frequency = df['Frequency'].sum()}:
با استفاده از این خط، فراوانی کلی تمام بازه‌های زمانی محاسبه می‌شود.

محاسبه فراوانی نسبی و توزیع تجمعی:

\lr{df['Relative Frequency'] = df['Frequency'] / total_frequency}:
این خط، فراوانی نسبی هر بازه زمانی را با تقسیم فراوانی هر بازه بر فراوانی کل محاسبه می‌کند.

\lr{df['Cumulative Distribution'] = df['Relative Frequency'].cumsum()}:
در این خط، توزیع تجمعی برای هر بازه زمانی را با جمع تراکمی فراوانی‌های نسبی محاسبه می‌کنیم.

تولید ۵ عدد تصادفی و تعیین بازه‌های زمانی متناظر:

\lr{random_numbers = np.random.uniform(0, 1, 5)}:
این خط، ۵ عدد تصادفی در بازه
$[0,1]$
تولید می‌کند.

تابع 
\lr{determine_service_time}:
این تابع برای هر عدد تصادفی R ، بازه زمانی متناظر با آن را با توجه به توزیع تجمعی مشخص می‌کند.

\lr{service_times = [determine_service_time(R, df) for R in random_numbers]}:
با استفاده از این خط، برای هر یک از اعداد تصادفی تولید شده، بازه زمانی متناظر را مشخص می‌کنیم.

در نهایت، بازه‌های زمانی متناظر با هر عدد تصادفی به همراه خود اعداد تصادفی چاپ می‌شوند. این اطلاعات می‌توانند برای تحلیل‌های بعدی یا برای تعیین زمان‌های سرویس در سیستم صف مورد استفاده قرار گیرند.