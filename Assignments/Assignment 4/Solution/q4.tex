\section*{سوال ۴}
	
داده‌های جدول زیر، نمونه‌های جمع‌آوری شده از مدت زمان سرویس در یک سیستم صف هستند. با کمک این داده‌ها، یک جدول برای تولید زمان‌های سرویس دهی ایجاد کنید (مشابه اسلاید ۱۶ از فصل ۷) و برای ۵ عدد تصادفی \( R_i \) زمان سرویس متناظر را تعیین کنید. برای تولید \( R_i \) می‌توانید از روشی دلخواه استفاده کنید.

\begin{center}
	\begin{tabular}{ |c|c| }
		\hline
		Interval (seconds) & Frequency \\
		\hline
		15-30 & 10 \\
		30-45 & 20 \\
		45-60 & 25 \\
		60-90 & 35 \\
		90-120 & 30 \\
		120-180 & 20 \\
		180-300 & 10 \\
		\hline
	\end{tabular}
\end{center}

\section*{جواب سوال ۴}

جدول تولید زمان‌های سرویس دهی بر اساس داده‌های جدول موجود:

\begin{center}
	\begin{tabular}{ |c|c|c|c| }
		\hline
		Interval (seconds) & Frequency & Relative Frequency & Cumulative Distribution \\
		\hline
		15-30 & 10 & 0.067 & 0.067 \\
		30-45 & 20 & 0.133 & 0.200 \\
		45-60 & 25 & 0.167 & 0.367 \\
		60-90 & 35 & 0.233 & 0.600 \\
		90-120 & 30 & 0.200 & 0.800 \\
		120-180 & 20 & 0.133 & 0.933 \\
		180-300 & 10 & 0.067 & 1.000 \\
		\hline
	\end{tabular}
\end{center}

برای پنج عدد تصادفی \( R_i \) تولید شده، بازه‌های زمانی متناظر به شرح زیر هستند:

\begin{itemize}
	\item \( R_1 = 0.693 \): بازه زمانی "90-120"
	\item \( R_2 = 0.376 \): بازه زمانی "60-90"
	\item \( R_3 = 0.380 \): بازه زمانی "60-90"
	\item \( R_4 = 0.706 \): بازه زمانی "90-120"
	\item \( R_5 = 0.516 \): بازه زمانی "60-90"
\end{itemize}
