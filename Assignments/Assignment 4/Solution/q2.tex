\section*{سوال ۲}

\textbf{سوال:} اعداد تصادفی زیر را در نظر بگیرید:
$$
0.05, 0.08, 0.14, 0.24, 0.33, 0.33, 0.39,
$$
$$
0.41, 0.44, 0.53, 0.56, 0.58, 0.63, 0.73,
$$
$$
 0.76, 0.83, 0.84, 0.88, 0.88, 0.93
$$
فرض اینکه این اعداد توزیع یکنواخت داشته باشند را با اعمال تست \textbf{Kolmogorov-Smirnov} در بخش الف (با سطح معناداری \(5\%\)) بررسی کنید.
و در نظر گرفتن \(10\) بازه برای اعداد تصادفی تولید شده، تست بخش قبل را این بار با روش \textbf{Chi-Square} در بخش ب (با سطح معناداری \(5\%\)) بررسی کنید.
در بخش ج، با توضیح دلیل تعیین کنید که از نتیجه کدام تست باید استفاده کرد.

\section*{جواب سوال ۲}

\textbf{الف) تست Kolmogorov-Smirnov:}

آماره تست: \(0.13\)

مقدار \(p\): \(0.846\)

با توجه به مقدار \(p\) بزرگتر از \(0.05\)، فرض توزیع یکنواخت رد نمی‌شود.

\textbf{ب) تست Chi-Square:}

آماره تست: \(5.0\)

مقدار \(p\): \(0.834\)

با توجه به مقدار \(p\) بزرگتر از \(0.05\)، فرض توزیع یکنواخت رد نمی‌شود.

\textbf{ج) انتخاب تست مناسب:}

با توجه به نتایج مشابه در هر دو تست، هر دو تست نشان می‌دهند که فرض توزیع یکنواخت برای این داده‌ها قابل قبول است. با این حال، انتخاب تست مناسب بستگی به شرایط خاص مطالعه و نوع داده‌ها دارد. تست

\lr{Kolmogorov-Smirnov}
برای مجموعه داده‌های کوچکتر مناسب است، در حالی که تست
\lr{Chi-Square}
برای مجموعه داده‌های بزرگتر با تقسیم‌بندی به بازه‌ها مفید است.