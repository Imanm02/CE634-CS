\section*{سوال ۲}

\subsection*{الف}

درباره صحیح یا غلط بودن بودن جملات زیر استدلال کنید (در صورت غلط بودن برای آن مثال نقض بیاورید)

\begin{itemize}
	\item \textbf{۱-} مدل‌های شبیه‌سازی گسسته را فقط برای مدل کردن سیستم‌های گسسته می‌توان استفاده کرد.
	\item \textbf{۲-} مدل‌های شبیه‌سازی پیوسته را فقط برای مدل کردن سیستم‌های پیوسته می‌توان استفاده کرد.
\end{itemize}

\subsection*{ب}

طبق اسلایدها می‌دانیم هنگامی که محیط تاثیر اندکی روی سیستم دارد، می‌توانیم به سه صورت آن را در شبیه‌سازی استفاده کنیم. این روش‌ها را شرح داده و برای هر یک مثالی که استفاده از آن روش مطلوب است را بیان کنید.

\section*{جواب سوال ۲}

\subsection*{الف}

\begin{itemize}
	\item \textbf{۱-} مدل‌های شبیه‌سازی گسسته را فقط برای مدل کردن سیستم‌های گسسته می‌توان استفاده کرد.
	\begin{itemize}
		\item \textbf{پاسخ:} این جمله \textbf{غلط} است. گاهی اوقات برای ساده‌سازی مدل‌سازی یک سیستم پیوسته، می‌توانیم از مدل‌های گسسته استفاده کنیم. مثلاً در مدل‌سازی جریان ترافیک که به طور واقعی پیوسته است، می‌توانیم از مدل‌های گسسته برای شبیه‌سازی حرکت خودروها در بازه‌های زمانی معین استفاده کنیم.
	\end{itemize}
	\item \textbf{۲-} مدل‌های شبیه‌سازی پیوسته را فقط برای مدل کردن سیستم‌های پیوسته می‌توان استفاده کرد.
	\begin{itemize}
		\item \textbf{پاسخ:} این جمله نیز \textbf{غلط} است. در برخی موارد، برای به دست آوردن نتایج دقیق‌تر و همچنین برای مدل‌سازی تغییرات گسسته با دقت بالا، می‌توان از مدل‌های شبیه‌سازی پیوسته استفاده کرد. به عنوان مثال، می‌توانیم سیستم صف گسسته را با استفاده از مدل پیوسته شبیه‌سازی کنیم تا نحوه تغییر تراکم صف در طول زمان را ببینیم.
	\end{itemize}
\end{itemize}

\subsection*{ب}

هنگامی که محیط تاثیر اندکی روی سیستم دارد، می‌توانیم به سه صورت آن را در شبیه‌سازی استفاده کنیم:

\begin{enumerate}
	\item \textbf{چشم‌پوشی از محیط:} در این روش، تاثیر محیط بر سیستم کاملاً نادیده گرفته می‌شود. 
	\begin{itemize}
		\item \textbf{مثال:} شبیه‌سازی فرآیند تولید در یک کارخانه که تاثیرات محیطی مانند دما یا رطوبت روی فرآیند تولید ناچیز است.
	\end{itemize}
	\item \textbf{تخمین تاثیر محیط:} در این روش، تاثیر محیط به صورت تخمینی و با استفاده از مقادیر میانگین یا ثابت در نظر گرفته می‌شود.
	\begin{itemize}
		\item \textbf{مثال:} شبیه‌سازی رشد گیاه در یک گلخانه، که تاثیر نور خورشید و دما روی رشد گیاه با استفاده از مقادیر میانگین در نظر گرفته می‌شود.
	\end{itemize}
	\item \textbf{انکلوژن محیط در مدل:} در این روش، محیط به صورت دقیق و با استفاده از داده‌ها و الگوریتم‌های مربوط به آن در مدل شبیه‌سازی گنجانده می‌شود.
	\begin{itemize}
		\item \textbf{مثال:} شبیه‌سازی جریان آب در یک رودخانه که عوامل محیطی مانند بارش باران، تغییرات فصلی و تغییرات زمین‌شناسی در مدل گنجانده می‌شود.
	\end{itemize}
\end{enumerate}